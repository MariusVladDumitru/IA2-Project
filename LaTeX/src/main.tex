\documentclass[journal, a4paper]{IEEEtran}

% some very useful LaTeX packages include:

%\usepackage{cite}      % Written by Donald Arseneau
                        % V1.6 and later of IEEEtran pre-defines the format
                        % of the cite.sty package \cite{} output to follow
                        % that of IEEE. Loading the cite package will
                        % result in citation numbers being automatically
                        % sorted and properly "ranged". i.e.,
                        % [1], [9], [2], [7], [5], [6]
                        % (without using cite.sty)
                        % will become:
                        % [1], [2], [5]--[7], [9] (using cite.sty)
                        % cite.sty's \cite will automatically add leading
                        % space, if needed. Use cite.sty's noadjust option
                        % (cite.sty V3.8 and later) if you want to turn this
                        % off. cite.sty is already installed on most LaTeX
                        % systems. The latest version can be obtained at:
                        % http://www.ctan.org/tex-archive/macros/latex/contrib/supported/cite/

\usepackage{graphicx}   % Written by David Carlisle and Sebastian Rahtz
                        % Required if you want graphics, photos, etc.
                        % graphicx.sty is already installed on most LaTeX
                        % systems. The latest version and documentation can
                        % be obtained at:
                        % http://www.ctan.org/tex-archive/macros/latex/required/graphics/
                        % Another good source of documentation is "Using
                        % Imported Graphics in LaTeX2e" by Keith Reckdahl
                        % which can be found as esplatex.ps and epslatex.pdf
                        % at: http://www.ctan.org/tex-archive/info/

%\usepackage{psfrag}    % Written by Craig Barratt, Michael C. Grant,
                        % and David Carlisle
                        % This package allows you to substitute LaTeX
                        % commands for text in imported EPS graphic files.
                        % In this way, LaTeX symbols can be placed into
                        % graphics that have been generated by other
                        % applications. You must use latex->dvips->ps2pdf
                        % workflow (not direct pdf output from pdflatex) if
                        % you wish to use this capability because it works
                        % via some PostScript tricks. Alternatively, the
                        % graphics could be processed as separate files via
                        % psfrag and dvips, then converted to PDF for
                        % inclusion in the main file which uses pdflatex.
                        % Docs are in "The PSfrag System" by Michael C. Grant
                        % and David Carlisle. There is also some information
                        % about using psfrag in "Using Imported Graphics in
                        % LaTeX2e" by Keith Reckdahl which documents the
                        % graphicx package (see above). The psfrag package
                        % and documentation can be obtained at:
                        % http://www.ctan.org/tex-archive/macros/latex/contrib/supported/psfrag/

%\usepackage{subfigure} % Written by Steven Douglas Cochran
                        % This package makes it easy to put subfigures
                        % in your figures. i.e., "figure 1a and 1b"
                        % Docs are in "Using Imported Graphics in LaTeX2e"
                        % by Keith Reckdahl which also documents the graphicx
                        % package (see above). subfigure.sty is already
                        % installed on most LaTeX systems. The latest version
                        % and documentation can be obtained at:
                        % http://www.ctan.org/tex-archive/macros/latex/contrib/supported/subfigure/

\usepackage{url}        % Written by Donald Arseneau
                        % Provides better support for handling and breaking
                        % URLs. url.sty is already installed on most LaTeX
                        % systems. The latest version can be obtained at:
                        % http://www.ctan.org/tex-archive/macros/latex/contrib/other/misc/
                        % Read the url.sty source comments for usage information.

%\usepackage{stfloats}  % Written by Sigitas Tolusis
                        % Gives LaTeX2e the ability to do double column
                        % floats at the bottom of the page as well as the top.
                        % (e.g., "\begin{figure*}[!b]" is not normally
                        % possible in LaTeX2e). This is an invasive package
                        % which rewrites many portions of the LaTeX2e output
                        % routines. It may not work with other packages that
                        % modify the LaTeX2e output routine and/or with other
                        % versions of LaTeX. The latest version and
                        % documentation can be obtained at:
                        % http://www.ctan.org/tex-archive/macros/latex/contrib/supported/sttools/
                        % Documentation is contained in the stfloats.sty
                        % comments as well as in the presfull.pdf file.
                        % Do not use the stfloats baselinefloat ability as
                        % IEEE does not allow \baselineskip to stretch.
                        % Authors submitting work to the IEEE should note
                        % that IEEE rarely uses double column equations and
                        % that authors should try to avoid such use.
                        % Do not be tempted to use the cuted.sty or
                        % midfloat.sty package (by the same author) as IEEE
                        % does not format its papers in such ways.

\usepackage{amsmath}    % From the American Mathematical Society
                        % A popular package that provides many helpful commands
                        % for dealing with mathematics. Note that the AMSmath
                        % package sets \interdisplaylinepenalty to 10000 thus
                        % preventing page breaks from occurring within multiline
                        % equations. Use:
%\interdisplaylinepenalty=2500
                        % after loading amsmath to restore such page breaks
                        % as IEEEtran.cls normally does. amsmath.sty is already
                        % installed on most LaTeX systems. The latest version
                        % and documentation can be obtained at:
                        % http://www.ctan.org/tex-archive/macros/latex/required/amslatex/math/



% Other popular packages for formatting tables and equations include:

%\usepackage{array}
% Frank Mittelbach's and David Carlisle's array.sty which improves the
% LaTeX2e array and tabular environments to provide better appearances and
% additional user controls. array.sty is already installed on most systems.
% The latest version and documentation can be obtained at:
% http://www.ctan.org/tex-archive/macros/latex/required/tools/

% V1.6 of IEEEtran contains the IEEEeqnarray family of commands that can
% be used to generate multiline equations as well as matrices, tables, etc.

% Also of notable interest:
% Scott Pakin's eqparbox package for creating (automatically sized) equal
% width boxes. Available:
% http://www.ctan.org/tex-archive/macros/latex/contrib/supported/eqparbox/

% *** Do not adjust lengths that control margins, column widths, etc. ***
% *** Do not use packages that alter fonts (such as pslatex).         ***
% There should be no need to do such things with IEEEtran.cls V1.6 and later.
\usepackage{siunitx}
\sisetup{output-exponent-marker=\ensuremath{\mathrm{e}}}


% Your document starts here!
\begin{document}

% Define document title and author
	\title{Charge to Mass Ratio of the Electron}
	\author{Jake Rugh \\ Physics 4L, Thursday 4:00pm}
	\maketitle

% Write abstract here
\begin{abstract}
	For an electron moving in a circular path in a magnetic field, if we know the magnetic field strength, accelerating voltage, and radius of the electron's trajectory, then we can make an estimation of the electron's charge to mass ratio. We calculated an average charge to mass ratio of $2.08 \times 10^{11} \pm 1.81 \times 10^8$ Coulombs per kilogram.
\end{abstract}

% Each section begins with a \section{title} command
\section{Theory}
	% \PARstart{}{} creates a tall first letter for this first paragraph
	\PARstart{T}{o} come up with a procedure to measure the charge to mass ratio of an electron, we had to find a way to relate the two quantities to each other mathematically. The first important relationship is between an electron's kinetic energy $K$ and total energy $E$.
    
    \begin{equation} \label{eq:Kinetic} % the label is used to reference the equation
K=\frac{1}{2}mv^2
\end{equation}

    \begin{equation} \label{eq:Energy} % the label is used to reference the equation
E=eV
\end{equation}

	$m$ is the electron's mass in kilograms, $v$ is the electron's velocity in meters per second, $e$ is the electron's charge in Coulombs, and $V$ is the accelerating voltage in Volts.

	When an electron is in motion, all if its energy is kinetic, and we can relate eq. \ref{eq:Kinetic} and eq. \ref{eq:Energy} in:

	\begin{equation} \label{eq:KtoE} % the label is used to reference the equation
\frac{1}{2}mv^2=eV
\end{equation}

	When the moving charged particle is subjected to a magnetic field, it experiences a force that is always perpendicular to its velocity $v$. The magnitude of this force $F$ can be calculated in:
    
    \begin{equation} \label{eq:MagForce} % the label is used to reference the equation
F=evB
\end{equation}

	$B$ is the strength of the magnetic field in Teslas. Because this force is always perpendicular to the particle's motion, it causes the particle to travel in a circular path. The force on a particle in a circular path can also be computed in:

	\begin{equation} \label{eq:CircForce} % the label is used to reference the equation
F=\frac{mv^2}{R}
\end{equation}

	$R$ is the radius of the circular path in meters. By setting eq. \ref{eq:MagForce} and eq. \ref{eq:CircForce} equal to each other, we find the following relationship:
    
    \begin{equation} \label{eq:MagtoCirc} % the label is used to reference the equation
evB=\frac{mv^2}{R}
\end{equation}
where we can solve for velocity $v$.
    
    \begin{equation} \label{eq:velocity} % the label is used to reference the equation
v=\frac{eBR}{m}
\end{equation}

	Substituting this value of $v$ into eq. \ref{eq:KtoE} allows us to find a way to directly calculate the charge to mass ratio of the electron.

\begin{equation} \label{eq:ratio} % the label is used to reference the equation
\frac{e}{m}=\frac{2V}{R^2B^2}
\end{equation}

	Therefore, all we need to know to find the charge to mass ratio of the electron undergoing circular motion in a magnetic field is the magnetic field strength $B$, the accelerating voltage $V$, and the radius of the electron's path $R$.

% Main Part
\section{Approach}
	The approach we used to measure the charge to mass ratio of the electron was designed after the experiment of Bainbridge \cite{Romblom}. The apparatus used consists of a vacuum tube supported between Helmholtz coils. A filament contained inside an anode with a single slit releases a thin beam of electrons into the vacuum tube, whose paths are made visible by mercury vapor in the vacuum tube. A power supply provides voltage to the anode and a DC Voltemeter attached across is used to record it. An Adjust-A-Volt is attached to the filament to control the current running through the filament. A separate power supply is used to power the Helmholtz coils.
    
    When the filament is heated by running current through it, electrons are evaporated out into a negatively charged "cloud" via thermionic emission. The anode, which is kept at a positive potential relative to the filament, attracts the electrons and accelerates them away from the filament. The potential difference between the anode and the filament provides the accelerating potential $V$ from eq. \ref{eq:ratio}.
    
    Once the electrons have escaped the anode, they are subjected to a magnetic field provided by the Helmholtz coils and travel in a circular path until they collide with a mercury atom. If the electron has enough kinetic energy, one of the mercury's electrons is ejected. When another electron takes its place, the excess energy given off is visible as blue light and makes it possible to see the electron beam. If the pressure is too high in the vacuum tubes, the electrons will only travel too small a distance before colliding with a particle to make a full beam. If the pressure is too low, too few collisions will take place to see the electron beam. At a pressure of about $10^{-4}$ atmospheres, the electrons can travel about 8 to 10 centimeters before colliding with another particle, and enough collisions occur to see the electron beam.
    
    Once the electron leaves the anode, all of its energy is kinetic and can be computed in eq. \ref{eq:Kinetic}. This allows us to use the relationship between kinetic energy and total energy shown in eq. \ref{eq:KtoE}. 
    
    To measure the charge to mass ratio of the electron, we need three measurements: the radius of the electrons' circular path $R$, the accelerating voltage $V$ of the anode, and the magnetic field strength $B$ provided by the Helmholtz coils.
    
    The radius of the electrons' path is measured by sight using measurement posts attached to the vacuum tube. The accelerating voltage is set using the anode power supply and recorded using the DC Voltemeter attached across. The magnetic field strength of the Helmholtz coil is computed from the equation for a magnetic field around current-carrying coil.
    
    \begin{equation} \label{eq:MagStrength} % the label is used to reference the equation
B=\frac{8\mu NI}{\sqrt{125}a}
\end{equation}

	$\mu$ is the permeability of free space, $4\pi  \times 10^{-7}$ Tesla-meters per square meter. $N$ is the number of turns in the Helmholtz coils, which in this case is 72. $a$ is the radius of the coils, which in this case is 0.33 meters. $I$ is the current running through the Helmholtz coils in Ampheres, which is set by the Helmholtz power supply. 
    
    By manipulating the strength of the magnetic field with the current running through the Helmholtz coils, we can change the radius of the electrons' path. We took 5 measurements for each of 3 different values of accelerating voltage in the anode: 25V, 30V, and 35V. At each value, we recorded the current through the Helmholtz coils needed to produce a circular path with each of the five measurement posts inside the vacuum tube. In total, this gave us 15 different sets of data. 

	% This is how you define a table: the [!hbt] means that LaTeX is forced (by the !) to place the table exactly here (by h), or if that doesnt work because of a pagebreak or so, it tries to place the table to the bottom of the page (by b) or the top (by t).
	\begin{table}[!hbt]
		% Center the table
		\begin{center}
		% Title of the table
		\caption{Initial Observations}
		\label{tab:InitialData}
		% Table itself: here we have two columns which are centered and have lines to the left, right and in the middle: |c|c|
		\begin{tabular}{|c|c|c|}
			% To create a horizontal line, type \hline
			\hline
			% To end a column type &
			% For a linebreak type \\
			Anode Voltage $V$ (V) & Radius $R$ (cm) & Helmholtz Current $I$ (A) \\
			\hline
			24.9 & 3.20 & 2.50 \\
			\hline
			24.9 & 3.90 & 2.08 \\
			\hline
			24.9 & 4.50 & 1.72 \\
			\hline
            24.9 & 5.20 & 1.51 \\
            \hline 
            24.9 & 5.70 & 1.30 \\
            \hline
            29.9 & 3.20 & 2.78 \\
            \hline
            29.9 & 3.90 & 2.27 \\
            \hline
            29.9 & 4.50 & 1.94 \\
            \hline
            29.9 & 5.20 & 1.66 \\
            \hline
            29.9 & 5.70 & 1.45 \\
            \hline
            35.0 & 3.20 & 3.03 \\
            \hline
            35.0 & 3.90 & 2.51 \\
            \hline
            35.0 & 4.50 & 2.09 \\
            \hline
            35.0 & 5.20 & 1.78 \\
           	\hline
            35.0 & 5.70 & 1.58 \\
            \hline
		\end{tabular}
		\end{center}
	\end{table}
    
    We assumed no uncertainty for $R$ as those values were given to us. For the uncertainty of $V$ and $I$, we assumed a constant uncertainty for the values given to us by the DC Voltemeter and power supply.
    
      \begin{equation} \label{eq:VUncertainty} % the label is used to reference the equation
\sigma V = \frac{0.1}{\sqrt{12}} V
\end{equation}

  \begin{equation} \label{eq:IUncertainty} % the label is used to reference the equation
\sigma I = \frac{0.1}{\sqrt{12}} A
\end{equation}
    
    Using the value of current through the Helmholtz coil at each data point, we then calculated the strength of the magnetic field using eq. \ref{eq:MagStrength} for each data point.
    
    \begin{table}[!hbt]
		% Center the table
		\begin{center}
		% Title of the table
		\caption{Magnetic Field Strength at Each Data Point}
		\label{tab:FieldStrength}
		% Table itself: here we have two columns which are centered and have lines to the left, right and in the middle: |c|c|
		\begin{tabular}{|c|c|c|}
			% To create a horizontal line, type \hline
			\hline
			% To end a column type &
			% For a linebreak type \\
			$V$ (V) & $R$ (cm) & Magnetic Field Strength $B$ (G) \\
			\hline
			24.9 & 3.20 & 4.90 \\
			\hline
			24.9 & 3.90 & 4.08 \\
			\hline
			24.9 & 4.50 & 3.37 \\
			\hline
            24.9 & 5.20 & 2.96 \\
            \hline 
            24.9 & 5.70 & 2.55 \\
            \hline
            29.9 & 3.20 & 5.45 \\
            \hline
            29.9 & 3.90 & 4.45 \\
            \hline
            29.9 & 4.50 & 3.80 \\
            \hline
            29.9 & 5.20 & 3.25 \\
            \hline
            29.9 & 5.70 & 2.84 \\
            \hline
            35.0 & 3.20 & 5.96 \\
            \hline
            35.0 & 3.90 & 4.92 \\
            \hline
            35.0 & 4.50 & 4.10 \\
            \hline
            35.0 & 5.20 & 3.49 \\
           	\hline
            35.0 & 5.70 & 3.10 \\
            \hline
		\end{tabular}
		\end{center}
	\end{table}
    
	For our uncertainty of the magnetic field, we found a constant error with eq. \ref{eq:BUncertainty}.
    
      \begin{equation} \label{eq:BUncertainty} % the label is used to reference the equation
\sigma B=(\frac{8\mu N}{\sqrt{125}a})\sigma I = 5.66 \times 10^{-3} G
\end{equation}

	With $V$, $R$, and $B$ recorded at each data point, we then calculated the charge to mass ratio and the uncertainty of the ratio at each data point.
    
    \begin{table}[!hbt]
		% Center the table
		\begin{center}
		% Title of the table
		\caption{Charge to Mass Ratio at Each Point}
		\label{tab:emRatio}
		% Table itself: here we have two columns which are centered and have lines to the left, right and in the middle: |c|c|
		\begin{tabular}{|c|c|c|c|}
			% To create a horizontal line, type \hline
			\hline
			% To end a column type &
			% For a linebreak type \\
			$V$ (V) & $R$ (cm) & $e/m$ (C/kg) & $\sigma e/m$ (C/kg) \\
			\hline
			24.9 & 3.20 & $2.01 \times 10^{11}$ & $5.18 \times 10^8$\\
			\hline
			24.9 & 3.90 & $1.97 \times 10^{11}$ & $5.93 \times 10^8$\\
			\hline
			24.9 & 4.50 & $2.16 \times 10^{11}$ & $7.69 \times 10^8$\\
			\hline
            24.9 & 5.20 & $2.10 \times 10^{11}$ & $8.40 \times 10^8$\\
            \hline 
            24.9 & 5.70 & $2.36 \times 10^{11}$ & $1.08 \times 10^9$\\
            \hline
            29.9 & 3.20 & $1.97 \times 10^{11}$ & $4.51 \times 10^8$\\
            \hline
            29.9 & 3.90 & $1.99 \times 10^{11}$ & $5.40 \times 10^8$\\
            \hline
            29.9 & 4.50 & $2.04 \times 10^{11}$ & $6.39 \times 10^8$\\
            \hline
            29.9 & 5.20 & $2.09 \times 10^{11}$ & $7.54 \times 10^8$\\
            \hline
            29.9 & 5.70 & $2.28 \times 10^{11}$ & $9.34 \times 10^8$\\
            \hline
            35.0 & 3.20 & $1.93 \times 10^{11}$ & $3.99 \times 10^8$\\
            \hline
            35.0 & 3.90 & $1.90 \times 10^{11}$ & $4.65 \times 10^8$\\
            \hline
            35.0 & 4.50 & $2.06 \times 10^{11}$ & $5.94 \times 10^8$\\
            \hline
            35.0 & 5.20 & $2.13 \times 10^{11}$ & $7.12 \times 10^8$\\
           	\hline
            35.0 & 5.70 & $2.25 \times 10^{11}$ & $8.42 \times 10^8$\\
            \hline
		\end{tabular}
		\end{center}
	\end{table}
    
    For the uncertainty of the charge to mass ratio, we used the following equation at each data point.
    
    \begin{equation} \label{eq:emUncertainty1} % the label is used to reference the equation
\sigma \frac{e}{m}=\sqrt{(\frac{d(\frac{e}{m})}{dB})^2\sigma B^2+\frac{d(\frac{e}{m})}{dV})^2\sigma V^2}
\end{equation}

	\begin{equation} \label{eq:emUncertainty2} % the label is used to reference the equation
\sigma \frac{e}{m}= \sqrt{(\frac{4V}{B^3R})^2\sigma B^2+(\frac{2}{B^2R^2})^2\sigma V^2}
\end{equation}

	We calculated an average charge to mass ratio of $2.08 \times 10^{11} \pm 1.81 \times 10^8$ Coulombs per kilogram.


\section{Conclusion}
	The accepted value for the charge to mass ratio of the electron is $1.76 \times 10^{11}$ Coulombs per kilogram compared to our value of $2.08 \times 10^{11} \pm 1.81 \times 10^8$ Coulombs per kilogram. The accepted value was not within the uncertainty of our calculated value. Our value was 15.3 percent away from the accepted value.
    
    I think our data was inaccurate partially because the equipment used was old and thus less precise. This experiment was heavily dependent upon measurements taken from the DC Voltemeter and the operation of the power supplies and Adjust-A-Volt. I think if we retried the experiment with newer, more precise equipment, our data would have been closer to the accepted value. 
    
    I also think our data was inaccurate because we used the equation for Newtonian kinetic energy in this experiment (eq. \ref{eq:Kinetic}) rather than using relativistic energy. According to special relativity, the actual total energy of a particle with mass is:
    
    \begin{equation} \label{eq:KRelativity} % the label is used to reference the equation
E=\gamma mc^2
\end{equation}
in which $\gamma$ is defined as:

	\begin{equation} \label{eq:gamma} % the label is used to reference the equation
\gamma = \frac{1}{\sqrt{1-\frac{v^2}{c^2}}}
\end{equation}

	$c$ is the speed of light, $3.00 \times 10^8$ meters per second, and $v$ is the velocity of the particle measured by a stationary observer. As $v$ approaches $c$, the value of $\gamma$ approaches infinity. 
    
    For a particle at rest, $\gamma$ simplifies to 1. That means that the energy of a particle at rest is given by:
    
    \begin{equation} \label{eq:restenergy} % the label is used to reference the equation
E=mc^2
\end{equation}

	Then, the kinetic energy of a particle is given by the total energy minus the rest energy:
    
    \begin{equation} \label{eq:realtivekinetic} % the label is used to reference the equation
K=(\gamma-1)mc^2
\end{equation}
    
    Thus, if the velocity of the electrons in the vacuum tube is sufficient, this true value of their kinetic energy could deviate from the value of the energy given by eq. \ref{eq:Kinetic}. This factor must be taken into account as a reason our measured value of the charge to mass ratio is different from the accepted value. 
    
    I also think our data was inaccurate because we took too few measurements. If we retried the experiment with more data points, our measured value would be more accurate.
    
\section{Acknowledgements}
	I would like to thank my lab partner Ian Dulchinos for his input and contribution to the measurements taken in this experiment. I would like to thank William Schultz for his guidance in this experiment. The equipment used was provided by Physics 4L laboratory fees at the University of California, Santa Barbara.

% Now we need a bibliography:
\begin{thebibliography}{1}

	%Each item starts with a \bibitem{reference} command and the details thereafter:
	\bibitem{Romblom} % Book
	Romblom, Dave and Robert Pizzi. {\em Physics 3L/4L Laboratory Manual}. Hayden-McNeil Publishing, USA, 4th edition, 2016.

\end{thebibliography}

% Your document ends here!
\end{document}